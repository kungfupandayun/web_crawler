\documentclass[french]{article}



% The default font size is 10pt; 11pt and 12pt are alternatives
\input{templates/structure.tex} % Include the document which specifies all packages and structural customizations for this template

\begin{document}


%----------------------------------------------------------------------------------------
%	HEADER IMAGE
%----------------------------------------------------------------------------------------

% Title
% -----
\JournalName{\Huge\textbf {LETTRE D'INFORMATION SUR LES AMENDES RGPD  }}
\noindent\HorRule{3pt} \\[-0.75\baselineskip]
\HorRule{1pt}

\vspace{0.5cm}
	\SepRule
\vspace{0.5cm}


%General Statistic
\NewsItem{\raggedright{\LARGE Amendes RGPD Sanctionnées en résumé}}
\justify
Le règlement n 2016/679, dit règlement général sur la protection des données, est un règlement de l'Union européenne qui constitue le texte de référence en matière de protection des données à caractère personnel. \\
\textbf{530} des amendes ont été infligées jusqu'à aujourd'hui.
Le total cumulé des amendes pour la protection des données s'élève désormais à \textbf{275 187 314€}.


\begin{figure}
	[H]\centering\includegraphics[width=0.7\linewidth]{graphs/top10_countries}
      \caption{Top 10 des pays de l'UE avec le plus grand nombre d'amendes}
\end{figure}
\begin{figure}
	[H]\centering\includegraphics[width=0.7\linewidth]{graphs/top10_countries_fines}
	\caption{Top 10 des pays de l'UE avec la somme d'amendes la plus élevée}
 \end{figure}


\newpage

	\begin{multicols}{2}
	\begin{figure}
		[H]\centering\includegraphics[width=1.0\linewidth]{graphs/sector_data}
		\caption{Top 10 des secteurs avec le plus grand nombre d'amendes}
	\end{figure}
	\begin{figure}
		[H]\centering\includegraphics[width=1\linewidth]{graphs/sector_data_fines}
		\caption{Top 10 des entreprises avec le plus grand nombre d'amendes}
	 \end{figure}
	
	\end{multicols}
	
	
	
	\begin{figure}
		[H]\centering\includegraphics[width=0.6\linewidth]{graphs/top10_controller}
		\caption{Top 10 des entreprises avec le plus grand nombre d'amendes}
	\end{figure}
	
	\begin{figure}
		[H]\centering\includegraphics[width=0.6\linewidth]{graphs/top10_controller_fines}
		\caption{Top 10 des entreprises avec le montant le plus élevé d'amendes}
	 \end{figure}



\newpage


	\begin{multicols}{2}
	\begin{figure}
		[H]\centering\includegraphics[width=1\linewidth]{graphs/top10_quoted} 
		\caption{Top 5 des articles cités du RGPD avec le plus grand nombre d'amendes}
	\end{figure}
	\begin{figure}
		[H]\centering\includegraphics[width=1\linewidth]{graphs/top10_quoted_fines} 
		\caption{Top 5 des articles  du RGPD avec le montant le plus élevé d'amendes}
	\end{figure}
	\end{multicols}
	
	
	\NewsItem{\raggedright{\LARGE Analyse sur l'application du RGPD au fil des ans}}
	\begin{figure}
		[H]\centering\includegraphics[width=0.8\linewidth]{graphs/acc_nb_cases_graph}
		\caption{Le nombre d'amendes (cumulatif) }
	\end{figure}
	
	\begin{multicols}{2}
	\begin{figure}
		[H]\centering\includegraphics[width=1.0\linewidth]{graphs/SumOfFinesperYear}
		\caption{La somme d'amende par an }
	 \end{figure}
\justify
	L'application du RGPD s'est accélérée depuis les premières années, tant sur le nombre de cas d'exécution que sur la somme des amendes. La sensibilisation du public et la couverture médiatique de la vie privée et de la protection des données se sont également accrues au fil des ans. Cependant, le RGPD risque d'échouer. Selon une étude de Brave, cela est dû au fait que les autorités de protection des données (DPA) ne donnent pas suffisamment de ressources humaines et financières pour accomplir leurs tâches. Seuls 6 DPA nationaux disposent de plus de 10 enquêteurs techniques spécialisés. La moitié de toutes les APD nationales reçoivent de petits budgets annuels (5 millions d'euros ou moins) de leurs gouvernements.
	\end{multicols}



\newpage



%Particular Year Statistic
\NewsItem{\raggedright{\LARGE Amendes RGPD Sanctionnées en 2019}}

	\begin{multicols}{2}
	
	En 2019,  il y a eu \textbf{165} amendes.
	Le total cumulé des amendes de protection des données pour 2019 se tient maintenant à \textbf{878 023 14€}.
	
	\begin{figure}[H]
	\centering\includegraphics[width=1\linewidth]{graphs/counter_year}
	\end{figure}


	Les amendes du RGPD pour 2019 se résume pour chaque mois comme suit :

	\begin{figure}
	[H]\centering\includegraphics[width = 1.2\linewidth]{graphs/NbFinesPerMonth_year_graph}
	\caption{Petit aperçu sur des mois 2019}
	\end{figure}

	\end{multicols}


	\begin{figure}
		[H]\centering\includegraphics[scale=.5]{graphs/top10_countries_year}
		\caption{Top 10 des pays de l'UE avec le plus grand nombre d'amendes en 2019}
	\end{figure}
	\begin{figure}
		[H]\centering\includegraphics[scale=.5]{graphs/top10_countries_year_fines}
		\caption{Top 10 des pays de l'UE avec la somme d'amendes la plus élevée en 2019}
	\end{figure}

\newpage
\justify
	\begin{multicols}{2}
	\heading{3 amendes les plus récentes en 2019}{3 pt}
		\begin{itemize}
			\item \textbf{2019-12-19} \newline
			150,000€ d'amende en GRÈCE pour Aegean Marine Petroleum Network Inc..
			\newline
			Des sociétés n'appartenant pas au groupe Aegean Marine Petroleum avaient accès à ses serveurs contenant des données personnelles et copiaient le contenu des serveurs, car Aegean Marine Petroleum n'avait pas pris les mesures techniques nécessaires pour sécuriser le traitement de grandes quantités de données et garder le logiciel concerné séparé
			\newline
			\href{http://www.dpa.gr/APDPXPortlets/htdocs/documentDisplay.jsp?docid=205,136,113,56,60,108,243,88}{Plus d'informations}
			\vspace{1cm}
	
			\item \textbf{2019-12-18} \newline
			2000€ d'amende en ROUMANIE pour Telekom Romania Mobile Communications SA.
			\newline
			La société n'a pas assuré l'exactitude du traitement des données personnelles qui a abouti à la divulgation des données personnelles d'un client à un autre client.
			\newline
			\href{https://www.dataprotection.ro/?page=O_noua_amenda_pentru_incalcarea_RGPD_comunicat_decembrie&lang=ro}{Plus d'informations}
			\vspace{1cm}
	
			\item \textbf{2019-12-17} \newline 15,000€ d'amende en BELGIQUE pour Site Web fournissant des informations juridiques
			\newline
			Un exploitant d'un site Web pour les nouvelles juridiques avait la déclaration de confidentialité disponible uniquement en anglais, bien qu'elle s'adresse également à un public néerlandophone et francophone.
			\newline
			\href{https://www.gegevensbeschermingsautoriteit.be/sites/privacycommission/files/documents/BETG_12-2019_NL.PDF}{Plus d'informations}
		\end{itemize}
	\end{multicols}

\newpage
\justify
	\begin{multicols}{2}
	\heading{Amendes notables en 2019}{3 pt}
		\begin{itemize}
			\item \textbf{La plus grosse amende} en 2019 - \textbf{50000000 €} a été condamné par FRANCE à Google Inc..
			\newline
			\textbf{Résumé} : L'amende a été infligée sur la base de plaintes de l'organisation autrichienne None Of Your Business et de l'ONG française La Quadrature du Net.
			\newline
			\href{https://www.cnil.fr/en/cnils-restricted-committee-imposes-financial-penalty-50-million-euros-against-google-llc}{Plus d'informations}
			\vspace{1cm}
		
			\item \textbf{La plus petite amende} en 2019 - \textbf{0 €} -  a été condamné par  à Poste autrichienne.
			\newline
			\textbf{Résumé} : Beau résumé original: La poste autrichienne avait créé des profils de plus de trois millions d'Autrichiens, qui comprenaient des informations sur leurs adresses personnelles, leurs préférences personnelles, leurs habitudes et leurs éventuelles affinités partisanes - qui ont ensuite été revendues, par exemple à des partis politiques et des entreprises. (Dans cette affaire, un jugement civil concernant des demandes d'indemnisation d'une valeur de 800 € a également été rendu: - cependant, cette décision de justice a déjà été annulée faute de preuves de dommages réels: )Mise à jour: Le Tribunal administratif fédéral a annulé l'amende de 18 millions d'euros infligée à la poste autrichienne. Le tribunal a généralement confirmé que le comportement de la poste autrichienne était illégal. Cependant, le tribunal a constaté une erreur formelle dans l'amende infligée à la poste autrichienne. L'autorité de protection des données aurait dû non seulement accuser la poste autrichienne de comportement illégal en tant que personne morale, mais également certaines personnes physiques.
			\newline
			\href{https://wien.orf.at/stories/3019396/}{Plus d'informations}
		\end{itemize}
	\end{multicols}


\newpage

	
	\begin{multicols}{2}
	\begin{figure}
		[H]\centering\includegraphics[width=1.0\linewidth]{graphs/top10_controller_year}
		\caption{Top 10 des entreprises avec le plus grand nombre d'amendes en2019}
	\end{figure}
	\begin{figure}
		[H]\centering\includegraphics[width=1\linewidth]{graphs/top10_controller_year_fines}
		\caption{Top 10 des entreprises avec le montant le plus élevé d'amendes en 2019}
	 \end{figure}
	
	\end{multicols}


		
	\begin{multicols}{2}
	\begin{figure}
		[H]\centering\includegraphics[width=1.0\linewidth]{graphs/sector_data_year_fines}
		\caption{Top 10 des secteurs avec le montant le plus élevé d'amendes2019}
	\end{figure}
	\begin{figure}
		[H]\centering\includegraphics[width=1\linewidth]{graphs/sector_data_year}
		\caption{Top 10 des secteurs avec le montant le plus élevé d'amendes2019}
	 \end{figure}
	
	\end{multicols}

	\begin{multicols}{2}
	\begin{figure}
		[H]\centering\includegraphics[width=1.0\linewidth]{graphs/top10_quoted_year}
		\caption{Top 5 des articles cités du RGPD avec le plus grand nombre d'amendes en 2019}
	\end{figure}
	\begin{figure}
		[H]\centering\includegraphics[width=1\linewidth]{graphs/top10_quoted_year_fines}
		\caption{Top 5 des articles  du RGPD avec le montant le plus élevé d'amendes en 2019}
	 \end{figure}
	
	\end{multicols}









\vspace*{\fill}
\textbf{Références:}\\
\href{https://www.enforcementtracker.com}{https://www.enforcementtracker.com}\\
\href{https://brave.com/wp-content/uploads/2020/04/Brave-2020-DPA-Report.pdf}{https://brave.com/wp-content/uploads/2020/04/Brave-2020-DPA-Report.pdf}\\
\href{https://arxiv.org/pdf/2011.00946.pdf}{https://arxiv.org/pdf/2011.00946.pdf}


\end{document}