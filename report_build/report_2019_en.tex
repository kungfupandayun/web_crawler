\documentclass[12pt]{article}


% The default font size is 10pt; 11pt and 12pt are alternatives
\input{templates/structure.tex} % Include the document which specifies all packages and structural customizations for this template

\begin{document}


%----------------------------------------------------------------------------------------
%	HEADER IMAGE
%----------------------------------------------------------------------------------------

% Title
% -----
\JournalName{\Huge\textbf {GDPR FINES NEWSLETTER}}
\noindent\HorRule{3pt} \\[-0.75\baselineskip]
\HorRule{1pt}

\vspace{0.5cm}
	\SepRule
\vspace{0.5cm}


%General Statistic
\NewsItem{\raggedright{\LARGE GDPR Fines Overview}}
\justify
The General Data Protection Regulation 2016/679 is a regulation in EU law on data protection and privacy in the European Union and the European Economic Area. \\
\textbf{530} fines have been given out until today.
The overall cumulative total of data protection fines now stands at \textbf{275 187 314€}.


\begin{figure}
	[H]\centering\includegraphics[width=0.7\linewidth]{graphs/top10_countries}
      \caption{Top 10 EU Countries with the highest number of fines }
\end{figure}
\begin{figure}
	[H]\centering\includegraphics[width=0.7\linewidth]{graphs/top10_countries_fines}
	\caption{Top 10 EU Countries with the highest sum of fines}
 \end{figure}


\newpage

	\begin{multicols}{2}
	\begin{figure}
		[H]\centering\includegraphics[width=1.0\linewidth]{graphs/sector_data}
		\caption{Top 10 Sectors on piechart with the highest number of fines}
	\end{figure}
	\begin{figure}
		[H]\centering\includegraphics[width=1\linewidth]{graphs/sector_data_fines}
		\caption{Top 10 Sectors with the highest amount of fines}
	 \end{figure}
	
	\end{multicols}
	
	
	
	\begin{figure}
		[H]\centering\includegraphics[width=0.6\linewidth]{graphs/top10_controller}
		\caption{Top 10 Companies with the highest number of fines}
	\end{figure}
	
	\begin{figure}
		[H]\centering\includegraphics[width=0.6\linewidth]{graphs/top10_controller_fines}
		\caption{Top 10 Companies with the highest amount of fines}
	 \end{figure}



\newpage


	\begin{multicols}{2}
	\begin{figure}
		[H]\centering\includegraphics[width=1\linewidth]{graphs/top10_quoted} 
		\caption{Top 5 Quoted Articles with the highest number of fines}
	\end{figure}
	\begin{figure}
		[H]\centering\includegraphics[width=1\linewidth]{graphs/top10_quoted_fines} 
		\caption{Top 5  Quoted Articles with the highest amount of fines}
	\end{figure}
	\end{multicols}
	
	
	\NewsItem{\raggedright{\LARGE Analysis on GDPR enforcement over years}}
	\begin{figure}
		[H]\centering\includegraphics[width=0.8\linewidth]{graphs/acc_nb_cases_graph}
		\caption{Overall Number of Fines (cumulative) }
	\end{figure}
	
	\begin{multicols}{2}
	\begin{figure}
		[H]\centering\includegraphics[width=1.0\linewidth]{graphs/SumOfFinesperYear}
		\caption{Overall Amount of Fines per year }
	 \end{figure}
\justify
	GDPR enforcement has accelerated since the first few years, both on the number of enforcement cases as well as the sum of fines. Public awareness and media coverage on privacy and data protection have increased over the years too. However, GDPR is at the risk of failing. According to a research from Brave, this is due to the Data Protection Authorities(DPA) not giving enough human and financial resources to perform their tasks. Only 6 national DPAs have more than 10 specialist tech investigation staff. Half of all national DPAs receive small (€5 million or less) annual budgets from their governments.
	\end{multicols}



\newpage



%Particular Year Statistic
\NewsItem{\raggedright{\LARGE GDPR fines overview in 2019}}

	\begin{multicols}{2}
	
	In 2019, there have been \textbf{165} fines.
	The cumulative total of data protection fines for 2019 now stands at \textbf{878 023 14€}.
	
	\begin{figure}[H]
	\centering\includegraphics[width=1\linewidth]{graphs/counter_year}
	\end{figure}


	The GPDR fines for 2019 break down for each months as follow :

	\begin{figure}
	[H]\centering\includegraphics[width = 1.2\linewidth]{graphs/NbFinesPerMonth_year_graph}
	\caption{Small Overview over months in 2019}
	\end{figure}

	\end{multicols}


	\begin{figure}
		[H]\centering\includegraphics[scale=.5]{graphs/top10_countries_year}
		\caption{Top 10 EU Countries with the highest number of fines in 2019}
	\end{figure}
	\begin{figure}
		[H]\centering\includegraphics[scale=.5]{graphs/top10_countries_year_fines}
		\caption{Top 10 EU Countries with the highest sum of fines in 2019}
	\end{figure}

\newpage
\justify
	\begin{multicols}{2}
	\heading{3 most recent fines in 2019}{3 pt}
		\begin{itemize}
			\item \textbf{2019-12-19} \newline
			150,000€ fine issued in GREECE to Aegean Marine Petroleum Network Inc..
			\newline
			Companies outside the Aegean Marine Petroleum Group had access to its servers containing personal data and copied the contents of the servers, since Aegean Marine Petroleum failed to take the necessary technical measures to secure the processing of large amounts of data and to keep the relevant software separate from the personal data stored on the servers. Furthermore, Aegean Marine Petroleum had not informed the data subjects of the processing of their personal data stored on the servers.
			\newline
			\href{http://www.dpa.gr/APDPXPortlets/htdocs/documentDisplay.jsp?docid=205,136,113,56,60,108,243,88}{More Info}
			\vspace{1cm}
	
			\item \textbf{2019-12-18} \newline
			2000€ fine issued in ROMANIA to Telekom Romania Mobile Communications SA.
			\newline
			The company has failed to ensure the accuracy of the processing of personal data which resulted in a disclosure of a clients personal data to another client.
			\newline
			\href{https://www.dataprotection.ro/?page=O_noua_amenda_pentru_incalcarea_RGPD_comunicat_decembrie&lang=ro}{More Info}
			\vspace{1cm}
	
			\item \textbf{2019-12-17} \newline 15,000€ fine issued in BELGIUM to Website providing legal information
			\newline
			An operator of a website for legal news had the privacy statement only available in English, although it was also addressed to a Dutch and French speaking audience. In addition, the first version of the privacy statement was not easily accessible and did not mention the legal basis for data processing under the GDPR. Furthermore, with reference to the ECJ ruling on Planet 49, it was determined that effective consent was required for the use of Google Analytics.
			\newline
			\href{https://www.gegevensbeschermingsautoriteit.be/sites/privacycommission/files/documents/BETG_12-2019_NL.PDF}{More Info}
		\end{itemize}
	\end{multicols}

\newpage
\justify
	\begin{multicols}{2}
	\raggedright\heading{Notable fines in 2019}{3 pt}
		\begin{itemize}
			\item The \textbf{largest} fine of 2019 - \textbf{50000000 €} - was issued in FRANCE to Google Inc..
			\newline
			\textbf{Summary} : The fine was imposed on the basis of complaints from the Austrian organisation None Of Your Business and the French NGO La Quadrature du Net. The complaints were filed on 25th and 28th of May 2018 - immediately after the GDPR became applicable. The complaints concerned the creation of a Google account during the configuration of a mobile phone using the Android operating system. The CNIL imposed a fine of 50 million euros for lack of transparency (Art. 5 GDPR), insufficient information (Art. 13 / 14 GDPR) and lack of legal basis (Art. 6 GDPR). The obtained consents had not been given specific and not unambigous (Art. 4 nr. 11 GDPR).
			\newline
			\href{https://www.cnil.fr/en/cnils-restricted-committee-imposes-financial-penalty-50-million-euros-against-google-llc}{More info}
			\vspace{1cm}
		
			\item The \textbf{lowest} fine of 2019 - \textbf{0 €} - was issued in  to Austrian Post.
			\newline
			\textbf{Summary} : Original Fine summary: The Austrian Post had created profiles of more than three million Austrians, which included information about their home addresses, personal preferences, habits and possible party affinity - which were subsequently resold, for example to political parties and companies. (In the case, also a civil court judgement about compensation claims at a value of 800 € has been issued: - however, this court decision has already been overturned due to lack of evidence of actual damage:)Update: The Federal Administrative Court overturned the fine of EUR 18 million imposed on the Austrian Post. The court generally confirmed that the behavior of the Austrian Post was illegal. However, the court found a formal error in the fine imposed on the Austrian Post. The data protection authority should have not only accused the Austrian Post of unlawful behavior as a legal entity, but also certain natural persons.
			\newline
			\href{https://wien.orf.at/stories/3019396/}{More info}
		\end{itemize}
	\end{multicols}


\newpage

	
	\begin{multicols}{2}
	\begin{figure}
		[H]\centering\includegraphics[width=1.0\linewidth]{graphs/top10_controller_year}
		\caption{Top 10 Companies with the highest number of fines in 2019}
	\end{figure}
	\begin{figure}
		[H]\centering\includegraphics[width=1\linewidth]{graphs/top10_controller_year_fines}
		\caption{Top 10 Companies with the highest  amount of fines in 2019}
	 \end{figure}
	
	\end{multicols}


		
	\begin{multicols}{2}
	\begin{figure}
		[H]\centering\includegraphics[width=1.0\linewidth]{graphs/sector_data_year_fines}
		\caption{Top 10 Sectors on piechart with the highest number of fines in 2019}
	\end{figure}
	\begin{figure}
		[H]\centering\includegraphics[width=1\linewidth]{graphs/sector_data_year}
		\caption{Top 10 Sectors with the highest amount of fines in 2019}
	 \end{figure}
	
	\end{multicols}

	\begin{multicols}{2}
	\begin{figure}
		[H]\centering\includegraphics[width=1.0\linewidth]{graphs/top10_quoted_year}
		\caption{Top 10 Quoted Article with the highest number of fines in 2019}
	\end{figure}
	\begin{figure}
		[H]\centering\includegraphics[width=1\linewidth]{graphs/top10_quoted_year_fines}
		\caption{Top 10 Quoted Article with the highest amount of fines in 2019}
	 \end{figure}
	
	\end{multicols}









\vspace*{\fill}
\textbf{References:}\\
\href{https://www.enforcementtracker.com}{https://www.enforcementtracker.com}\\
\href{https://brave.com/wp-content/uploads/2020/04/Brave-2020-DPA-Report.pdf}{https://brave.com/wp-content/uploads/2020/04/Brave-2020-DPA-Report.pdf}\\
\href{https://arxiv.org/pdf/2011.00946.pdf}{https://arxiv.org/pdf/2011.00946.pdf}


\end{document}